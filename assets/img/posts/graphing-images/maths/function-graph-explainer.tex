\documentclass[preview]{standalone}
\usepackage{amsmath}
\begin{document}

Look at the behaviour of $f(x) = y$ when we fix a $y$ and
investigate the acceptable range of values $x$ can take while
still being within the line of the graph.
First fix $\Delta$ to be the cutoff value where we set a pixel to on
given the following holds

\[
  |f(x) - y| < \Delta
\]

When we fix $y = b$ for some constant $b$, then for
$x$ satisfying
\[
  b - \Delta < f(x) < b + \Delta
\]
The pixel is on. So this gives a range of $2\Delta$ for values of
$f(x)$. So that the range of values for $x$ will often vary along the
length of the function.

For a particular example look at the function $f(x)=x^3$ which has an
inverse function $x^{\frac{1}{3}}$.
\begin{align*}
  b - \Delta &< x^3 < b + \Delta \\
 (b - \Delta)^{\frac{1}{3}} &< x < (b + \Delta)^{\frac{1}{3}}
\end{align*}

Giving values of x a range of
\[
  |(b - \Delta)^{\frac{1}{3}} - (b + \Delta)^{\frac{1}{3}}|
\]

Which gives a large cutoff for $b$ close to zero and a smaller cutoff as $b$
increases. 
\end{document}